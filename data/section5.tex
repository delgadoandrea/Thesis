%%%%%%%%%%%%%%%%%%%%%%%%%%%%%%%%%%%%%%%%%%%%%%%%%%%
%
%  New template code for TAMU Theses and Dissertations starting Fall 2016.  
%
%
%  Author: Sean Zachary Roberson
%  Version 3.17.09
%  Last Updated: 9/21/2017
%
%%%%%%%%%%%%%%%%%%%%%%%%%%%%%%%%%%%%%%%%%%%%%%%%%%%
%%%%%%%%%%%%%%%%%%%%%%%%%%%%%%%%%%%%%%%%%%%%%%%%%%%%%%%%%%%%%%%%%%%%%%
%%                           SECTION V
%%%%%%%%%%%%%%%%%%%%%%%%%%%%%%%%%%%%%%%%%%%%%%%%%%%%%%%%%%%%%%%%%%%%%



\chapter{ANALYSIS \label{cha:analysis}}

%Here an overview of the strategy, decribe signal generaly, what does it look like. Then the background, how to discriminate s/b to motivate following subsections.
This study presents the search for a heavy boson resonance decaying into two $b$ quarks. Suchs prticle arising from the fusion of two $b$ quarks which are in turn a result of gluon splitting. The final state then consists of four $b$ quarks, two of them being direct decay products of the resonance and two additional associated quarks from the initial gluon splitting.

Any signal for the process of interest would be buried under an inmense background of multi-jet events produced by QCD interactions and this analysis seeks to detect it using $b$-tagging and a data-driven estimation of the background.

Data for this analysis were collected using a HLT that uses multi-jet and $b$-tagging requirements as described in Section. The Z' decay is reconstructed offline through PF jets clustered using the anti-$k_{T}$ algorithm with a size parameter of 0.4 that are tagged by the DeepCSV algorithm to be consisten with $b$-jets. This is described in Section. A signal region is identified by means of a specific $b$-tagging selection on the 4 leading jets on $p_{T}$ and an offline trigger which will be described in Section. The background control regions are orthogonal to the SR in terms of the trigger and $b$-tagging selectrion used for the SR. Then, a data-driven method is used to estimate the background contribution due to the poorly-understood multi-jet QCD backgrounds. 

Once the contribution from undertainties is well understood and the data-driven method is alidated, the SR will be unblinded and the limits on the production cross section are calculated.

\section{Data and Monte Carlo Samples}
\subsection{Data}

\begin{table}[hbtp]\footnotesize
\centering
\begin{tabular}{l l l}
\hline
 Dataset & Run Range & Integrated Luminosity \\
\hline
/BTagCSV/Run2016B-07Aug17-v*/AOD & 272007-275376 & 5.8$\fbinv$ \\
/BTagCSV/Run2016C-07Aug17-v1/AOD & 275657-276283 & 2.5$\fbinv$ \\
/BTagCSV/Run2016D-07Aug17-v1/AOD & 276315-276811 & 4.3$\fbinv$ \\
/BTagCSV/Run2016E-07Aug17-v1/AOD & 276831-277420 & 4.1$\fbinv$ \\
/BTagCSV/Run2016F-07Aug17-v1/AOD & 277772-278808 & 3.1$\fbinv$ \\
/BTagCSV/Run2016G-07Aug17-v1/AOD & 278820-280385 & 7.5$\fbinv$ \\
/BTagCSV/Run2016H-07Aug17-v1/AOD & 280919-284044 & 8.5$\fbinv$ \\
{\bf Total BTagCSV} & {\bf 272007--284044} & {\bf 35.9$\fbinv$} \\
\hline
/SingleMu/Run2016B-07Aug17-v*/AOD & 272007-275376 & 5.8$\fbinv$ \\
/SingleMu/Run2016C-07Aug17-v1/AOD & 275657-276283 & 2.5$\fbinv$ \\
/SingleMu/Run2016D-07Aug17-v1/AOD & 276315-276811 & 4.3$\fbinv$ \\
/SingleMu/Run2016E-07Aug17-v1/AOD & 276831-277420 & 4.1$\fbinv$ \\
/SingleMu/Run2016F-07Aug17-v1/AOD & 277772-278808 & 3.1$\fbinv$ \\
/SingleMu/Run2016G-07Aug17-v1/AOD & 278820-280385 & 7.5$\fbinv$ \\
/SingleMu/Run2016H-07Aug17-v1/AOD & 280919-284044 & 8.5$\fbinv$ \\
{\bf Total SingleMu} & {\bf 272007--284044} & {\bf 35.9$\fbinv$} \\
\hline
\end{tabular}
\caption{The datasets analyzed for this analysis.}
\label{tab:dataSamples}
\end{table}

Table \ref{tab:dataSamples} summarizes the $\sqrt{s}$= 13 TeV datasets from data collected during 2016 used in this analysis. Data is processed in the CMSSW_9_4_9 framework to yield \textit{n-tuples} using a subset of the information contained in the official CMS datasets and was anlyzed using privately created analysis software.

\subsection{Monte Carlo}

MC simulations are used to provide: predictions of background processes, optimization of the event selection, and cross-checks of data-based background estimations.

Signal and QCD background events are generated using the leading-order matrix element generator MADGRAPH 5v. Parton shower and hadronization are included using PYTHIA 6.4.26, and the matrix elemetn is matched to the parton shower using the MLM scheme. The Z2* tune is used to describe the underlying event. This tune is identical to the Z1 tune, but uses the CTEQ6L PDFs.

The signal events are simulated...

Different samples are generated for $m_{X}$ ranging from 250 GeV to 1 TeV.

All generated events are processed through a simulation of the CMS apparatus based on GEANT4. Additional proton-proton interactions within a bunch crossing(pileup) are added to the simulation, with a frequency distribution chosen to match that observed in data. During this data-taking period the mean number of interactions per bunch crossing is 25.

The analysis is based on data from proton-proton interactions observed with the CMS detector at $sqrt{s}$ = 13 TeV. The data corresponds to an integrated luminosity of 35.9 $fb^{-1}. Events are collected using at least one of the two specific trigger conditions based on jets reconstructed online:


\section{Event and Object Selection}

The event selection begins with identifying events containing at least 4 central jets with a $p_{T}>$ 30 GeV. Two of these jets (the most energetic ones) are expected to be associated with the resonance and therefore their reconstructed invariant mass is used as the discriminant variable.

\section{Trigger}
\subsection{Trigger Efficiency Estimation}
\section{Data-driven Background Estimation}
\section{Systematic Uncertainties}




