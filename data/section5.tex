%%%%%%%%%%%%%%%%%%%%%%%%%%%%%%%%%%%%%%%%%%%%%%%%%%%
%
%  New template code for TAMU Theses and Dissertations starting Fall 2016.  
%
%
%  Author: Sean Zachary Roberson
%  Version 3.17.09
%  Last Updated: 9/21/2017
%
%%%%%%%%%%%%%%%%%%%%%%%%%%%%%%%%%%%%%%%%%%%%%%%%%%%
%%%%%%%%%%%%%%%%%%%%%%%%%%%%%%%%%%%%%%%%%%%%%%%%%%%%%%%%%%%%%%%%%%%%%%
%%                           SECTION V
%%%%%%%%%%%%%%%%%%%%%%%%%%%%%%%%%%%%%%%%%%%%%%%%%%%%%%%%%%%%%%%%%%%%%



\chapter{ANALYSIS \label{cha:analysis}}

%Here an overview of the strategy, decribe signal generaly, what does it look like. Then the background, how to discriminate s/b to motivate following subsections.

The mass and b flavour properties of the leading jets are used to suppress the multijet QCD background.

\section{Data and Monte Carlo Samples}

MC simulations are used to provide: predictions of background processes, optimization of the event selection, and cross-checks of data-based background estimations.

Signal and QCD background events are generated using the leading-order matrix element generator MADGRAPH 5v. Parton shower and hadronization are included using PYTHIA 6.4.26, and the matrix elemetn is matched to the parton shower using the MLM scheme. The Z2* tune is used to describe the underlying event. This tune is identical to the Z1 tune, but uses the CTEQ6L PDFs.

The signal events are simulated...

Different samples are generated for $m_{X}$ ranging from 250 GeV to 1 TeV.

All generated events are processed through a simulation of the CMS apparatus based on GEANT4. Additional proton-proton interactions within a bunch crossing(pileup) are added to the simulation, with a frequency distribution chosen to match that observed in data. During this data-taking period the mean number of interactions per bunch crossing is 25.

The analysis is based on data from proton-proton interactions observed with the CMS detector at $sqrt{s}$ = 13 TeV. The data corresponds to an integrated luminosity of 31.31 $fb^{-1}. Events are collected using at least one of the two specific trigger conditions based on jets reconstructed online:


\section{Event and Object Selection}
\section{Trigger}
\subsection{Trigger Efficiency Estimation}
\section{Data-driven Background Estimation}
\section{Systematic Uncertainties}




